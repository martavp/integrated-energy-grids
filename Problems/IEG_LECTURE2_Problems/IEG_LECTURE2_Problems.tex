\documentclass[10pt]{article}
\usepackage{graphicx} % Required for inserting images
\usepackage{url}
\usepackage{hyperref}
\title{IEG_Problems_Lecture3}
\author{Aleksander Grochowicz}
\date{January 2025}

\usepackage[margin=1in]{geometry} 
\usepackage{amsmath,amsthm,amssymb, graphicx, multicol, array}
 
\newcommand{\N}{\mathbb{N}}
\newcommand{\Z}{\mathbb{Z}}
 
\newenvironment{problem}[2][Problem]{\begin{trivlist}
\item[\hskip \labelsep {\bfseries #1}\hskip \labelsep {\bfseries #2.}]}{\end{trivlist}}

\begin{document}
 
\title{\textbf{Lecture 2: Optimization}}
\author{
%Your name\\
DTU Course 46770: Integrated Energy Grids }
\maketitle
\begin{problem}{2.1}

Given the following optimization problem,


\begin{align*}
	\max_{x,y} \quad & 2x + 5y \\
	\text{s.t.} \quad & -x + y \geq -3, \\
	& 2x + y \leq 14, \\
	& -\frac{1}{2}x + y \geq \frac{3}{2}, \\
	& x \geq 1.
\end{align*}



\begin{itemize}
\item[a)] Solve the optimization problem graphically (pen and paper or on your laptop). Note that it is a maximization problem, whereas we will mostly work with minimizations. Reformulate as a minimization problem.
\item[b)] Return to the original formulation. Indicate which constraints are binding and calculate the values of the Lagrange multipliers.
\item[c)] In the above set-up, we have a unique solution to our maximization problem (existence and uniqueness). Adapt the exercise such that this is no longer the case.
\end{itemize}
\end{problem}

\

\begin{problem}{2.2}
	
	Consider the following economic dispatch problem: 
	\begin{itemize}
		\item we have three generators: solar, wind and gas
		\item solar and wind have no marginal costs, and gas has fuel costs of 60 EUR/MWh.
		\item we need to cover demand of 13.2 MWh
		\item the installed capacities are 15 MW, 20 MW and 20 MW for wind, solar, and gas, respectively
		\item assume the capacity factor for solar is 0.17 and for wind 0.33.
	\end{itemize}

	
	\begin{itemize}
		\item[a)] Use linopy to define and solve the LP and find the optimal solution as well as reading out the Lagrange multipliers as defined in the lecture. 
		\item[b)] Open \href{problem2b.csv}{\texttt{problem2b.csv}}, and use the values as inputs for capacity factors as well as demand in the dispatch problem. Solve the LP with linopy.
		\item[c)] Open \href{problem2c.csv}{\texttt{problem2c.csv}}, and use the values as inputs for capacity factors as well as demand in the dispatch problem. Solve the LP with linopy.
		\item[d)] Compare the share of renewable generation, the dual variables, and the objective from a)-c) [average, median, min, max] and interpret the differences. Compute the curtailment from renewables.
		\item[e)] Plot the supply and demand curves for the different resources in c). Also consider \texttt{demand - renewable generation} (``net load''). Could transmission or storage be useful for this system? Why or why not?
	\end{itemize}

\end{problem}

\

\begin{problem}{2.3}
	
	We assume the system from Problem 2.2.
	\begin{itemize}
		\item[a)] Another system that is connected to ours --- assuming copper-plate --- decides dispatch just before we do and can export utility solar. The results from that dispatch optimization are saved in \href{problem3a.csv}{\texttt{problem3a.csv}}. Create a time series of available imports and their price.
		\item[b)] Solve the updated problem with the available imports. Think of the imports as a generator with variable, marginal prices corresponding to the dual variables.
		% \item[c)] Now formulate the two problems as a joint optimization (so, one common demand, and combining all generators), what difference do you see to the previous solutions? Try to interpret what happened.
	\end{itemize}

\end{problem}

\

%\begin{proof}[Solution]
%Write a solution here
%\end{proof}

\end{document}


 

